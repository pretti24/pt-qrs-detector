
%TP genérico



%Configuracion del documento

%Tamaño de letra principal:

\documentclass[12pt]{article}


%Título y autor(es):

\title{Título del TP}

\author{Sergio}


%Tamaño de página y los márgenes:

\usepackage[a4paper,headheight=16pt]{geometry}
\textwidth      =  450pt     %Ancho del cuerpo
\textheight     =  548pt     %Largo del cuerpo
\topmargin      =  0pt       %Agrega espacio en el margen superior
\oddsidemargin  =  0pt       %+ margen izquierdo en paginas impares
\evensidemargin =  0pt       %+ margen derecho en paginas impares


% Vamos a escribir en castellano:

\usepackage[spanish]{babel}

%instalar texlive-lang-spanish
% Reconocer acentos y caracteres no ingleses:

\usepackage[utf8]{inputenc}


% Cabecera y pie de página personalizadas

\usepackage{fancyhdr}

\pagestyle{fancy}



% Hago que en la cabecera de página se muestre a la derecha la sección,
% y en el pie, en número de página a la derecha:
\renewcommand{\sectionmark}[1]{\markboth{}{\thesection\ \ #1}}
\lhead{}
\chead{}
\rhead{\rightmark}
\lfoot{}
\cfoot{}
\rfoot{\thepage}


%numeracion especial para tablas, figuras y ecuaciones

\usepackage{amssymb,amsmath}
\numberwithin{equation}{section}
\numberwithin{figure}{section}
\numberwithin{table}{section}


%Agregar notas al pie en tablas:
\usepackage{threeparttable}


%Incluir Graficos
\usepackage{graphicx}


%Usar subfiguras: (al estilo Figura 2.3(b) )
\usepackage{subfigure}
% Para esto es necesario texlive-latex-recommended o texlive-latex-extra


%Numero de figuras en negrita

\usepackage[hang,bf]{caption}


% Todas las imágenes están en el directorio tp-img:

\newcommand{\imgdir}{tp-img}
\graphicspath{{\imgdir/}}

% uso de colores
\usepackage{color}


%Para embeber código de lenguajes como Matlab, C, html, etc.

\usepackage{listings}
\lstset{ frame=Ltb, 
	framerule=0pt, 
	aboveskip=0.5cm, 
	framextopmargin=3pt, 
	framexbottommargin=3pt, 
	framexleftmargin=0.4cm, 
	framesep=0pt, rulesep=.4pt, 
	backgroundcolor=\color{gray}, 
	rulesepcolor=\color{black}, 
	showstringspaces = false, 
	basicstyle=\small\ttfamily, 
	commentstyle=\color{gray}, 
	keywordstyle=\bfseries
	}


	%rulesepcolor=\color{black},
	%stringstyle=\ttfamily,
	%showstringspaces = false,
	%basicstyle=\small\ttfamily,
	%commentstyle=\color{gray},
	%keywordstyle=\bfseries,

	%numbers=left,
	%numbersep=15pt,
	%numberstyle=\tiny,
	%numberfirstline = false,
	%breaklines=true


%Para colocar hipervinculos
\usepackage[dvipdfm,colorlinks=true,linkcolor=black,urlcolor=blue]{hyperref}


%Para hacer matrices y digramas de flujo sencillos
%\usepackage[all]{xy}


\usepackage{pdfpages}



%INICIO DEL DOCUMENTO
%-------------------------------------------------------------------------------
\begin{document}


% CARATULA

%-------------------------------------------------------------------------------


\begin{titlepage}
% Sin cabecera ni pie de página:

\thispagestyle{empty}

% Logo de la facu: 
\begin{center}
    \includegraphics[width=10cm]{logo-facu}
\end{center}

\vfill

% Título:
\begin{center}
	\Huge{Monitor HRV en tiempo real}\\
	\Large{Introducción a Proyectos - TP3}\\
	\Large{Planeamiento}
\end{center}

\vspace{4cm}

% Integrantes:

\large{
	\begin{tabbing}
		Sergio Hinojosa \hspace{1cm} \\84476 \\
	\end{tabbing}
}

\vfill

% Fecha o cuatrimestre:

\flushright{2011}

\end{titlepage}


% Hago que las páginas se comiencen a contar a partir de aquí:
\setcounter{page}{1}


%INDICE

%-------------------------------------------------------------------------------

% Pongo el índice en una página aparte:

%\tableofcontents

%\newpage



%-------------------------------------------------------------------------------
%incluir un pdf
%\includepdf{tareas.pdf} 

% Inicio del TP:
\section{Objetivo}

\begin{equation}
\begin{split}
H(z) = & \frac{1}{8T}(\frac{-z^2}{32}-\frac{5z}{32}-\frac{5z^{-1}}{8}-\frac{7z^{-2}}{8}-\frac{9z^{-3}}{8}-\frac{21z^{-4}}{16}-\frac{21z^{-5}}{16}-\frac{9z^{-6}}{8}-\frac{7z^{-7}}{8}-\frac{5z^{-8}}{8}-\\
&-\frac{3z^{-9}}{8}-\frac{5z^{-10}}{32}-\frac{z^{-11}}{32}+z^{-14}+4z^{-15}+7z^{-16}+8z^{-17}+8z^{-18}+8z^{-19}+6z^{-20}-\\
&-6z^{-22}-8z^{-23}-8z^{-24}-8z^{-25}-7z^{-26}-4z^{-27}-z^{-28}+\frac{z^{-30}}{32}+\frac{5z^{-31}}{32}+\frac{3z^{-32}}{8}+\\
&+\frac{5z^{-33}}{8}+\frac{7z^{-34}}{8}+\frac{9z^{-35}}{8}+\frac{21z^{-36}}{16}+\frac{21z^{-37}}{16}+\frac{9z^{-38}}{8}+\frac{7z^{-39}}{8}+\frac{5z^{-40}}{8}+\frac{3z^{-41}}{8}+\\
&+\frac{5z^{-42}}{32}+\frac{z^{-43}}{32}-\frac{3}{8})
\end{split}
\end{equation}

\begin{equation}
\begin{split}
H(z) z^{-2}= & \frac{1}{8T}(\frac{-1}{32}-\frac{5z^{-1}}{32}-\frac{5z^{-3}}{8}-\frac{7z^{-4}}{8}-\frac{9z^{-5}}{8}-\frac{21z^{-6}}{16}-\frac{21z^{-7}}{16}-\frac{9z^{-8}}{8}-\frac{7z^{-9}}{8}-\frac{5z^{-10}}{8}-\\
&-\frac{3z^{-11}}{8}-\frac{5z^{-12}}{32}-\frac{z^{-13}}{32}+z^{-16}+4z^{-17}+7z^{-18}+8z^{-19}+8z^{-20}+8z^{-21}+6z^{-22}-\\
&-6z^{-24}-8z^{-25}-8z^{-26}-8z^{-27}-7z^{-28}-4z^{-29}-z^{-30}+\frac{z^{-32}}{32}+\frac{5z^{-33}}{32}+\frac{3z^{-34}}{8}+\\
&+\frac{5z^{-35}}{8}+\frac{7z^{-36}}{8}+\frac{9z^{-37}}{8}+\frac{21z^{-38}}{16}+\frac{21z^{-39}}{16}+\frac{9z^{-40}}{8}+\frac{7z^{-41}}{8}+\frac{5z^{-42}}{8}+\frac{3z^{-43}}{8}+\\
&+\frac{5z^{-44}}{32}+\frac{z^{-45}}{32}-\frac{3z^{-2}}{8})
\end{split}
\end{equation}

\begin{equation}
\begin{split}
H(z) z^{-2}= & \frac{1}{8T}(\frac{-1}{32}-\frac{5z^{-1}}{32}-\frac{5z^{-3}}{8}-\frac{7z^{-4}}{8}-\frac{9z^{-5}}{8}-\frac{21z^{-6}}{16}-\frac{21z^{-7}}{16}-\frac{9z^{-8}}{8}-\frac{7z^{-9}}{8}-\frac{5z^{-10}}{8}-\\
&-\frac{3z^{-11}}{8}-\frac{5z^{-12}}{32}-\frac{z^{-13}}{32}+z^{-16}+4z^{-17}+7z^{-18}+8z^{-19}+8z^{-20}+8z^{-21}+6z^{-22}-\\
&-6z^{-24}-8z^{-25}-8z^{-26}-8z^{-27}-7z^{-28}-4z^{-29}-z^{-30}+\frac{z^{-32}}{32}+\frac{5z^{-33}}{32}+\frac{3z^{-34}}{8}+\\
&+\frac{5z^{-35}}{8}+\frac{7z^{-36}}{8}+\frac{9z^{-37}}{8}+\frac{21z^{-38}}{16}+\frac{21z^{-39}}{16}+\frac{9z^{-40}}{8}+\frac{7z^{-41}}{8}+\frac{5z^{-42}}{8}+\frac{3z^{-43}}{8}+\\
&+\frac{5z^{-44}}{32}+\frac{z^{-45}}{32}-\frac{3z^{-2}}{8})
\end{split}
\end{equation}

\begin{equation}
y(z) z^{-2} = x(z)H(z) z^{-2}
\end{equation}

\begin{equation}
\begin{split}
y(n-2)= & \frac{1}{8T}(\frac{-x(n)}{32}-\frac{5x(n-1)}{32}-\frac{5x(n-3)}{8}-\frac{7x(n-4)}{8}-\frac{9x(n-5)}{8}-\frac{21x(n-6)}{16}-\\
&-\frac{21x(n-7)}{16}-\frac{9x(n-8)}{8}-\frac{7x(n-9)}{8}-\frac{5x(n-10)}{8}-\frac{3x(n-11)}{8}-\\
&-\frac{5x(n-12)}{32}-\frac{x(n-13)}{32}+x(n-16)+4x(n-17)+7x(n-18)+\\
&+8x(n-19)+8x(n-20)+8x(n-21)+6x(n-22)-6x(n-24)-8x(n-25)-\\
&-8x(n-26)-8x(n-27)-7x(n-28)-4x(n-29)-x(n-30)+\\
&+\frac{x(n-32)}{32}+\frac{5x(n-33)}{32}+\frac{3x(n-34)}{8}+\frac{5x(n-35)}{8}+\frac{7x(n-36)}{8}+\\
&+\frac{9x(n-37)}{8}+\frac{21x(n-38)}{16}+\frac{21x(n-39)}{16}+\frac{9x(n-40)}{8}+\frac{7x(n-41)}{8}+\\
&+\frac{5x(n-42)}{8}+\frac{3x(n-43)}{8}+\frac{5x(n-44)}{32}+\frac{x(n-45)}{32}-\frac{3x(n-2)}{8})
\end{split}
\end{equation}

\section{Introducción}

\section{Tareas a realizar}

\section{Duración de la tareas}

\begin{itemize}
\item Tiempo optimista ($t_o$). Es el tiempo mínimo en que se pudiere realizar
  el proceso (en las mejores condiciones) cumpliendo con lo mínimo establecido.
\item Tiempo pesimista ($t_p$). Indica el mayor tiempo que se considera que
  pudiere demandar la tarea bajo las peores condiciones.
\item Tiempo más probable ($t_m$). Es el tiempo que lleva la tarea en
  condiciones normales de realización.
\end{itemize}

%\begin{figure}[htb]
%	\centering
%	\includegraphics[width=8cm]{gant}
	%\caption{}
	%\label{fig:figura1}
%\end{figure}

\subsection{Pert}


[1]\href{http://www.conicet.gov.ar/scp/search.php?page=1&keywords=ecg}{http://www.conicet.gov.ar/scp/search.php?page=1\&keywords=ecg}.

\subsection{Ciclo de vida del producto}


\begin{tabular}{c | c | c | c | c | c | c | c | c | c | c | c | c | c | c | c | c | }
   Mes & 1 & 2 & 4 & 6 & 8 & 10 & 12 & 14 & 16 & 18 & 20 & 22 & 24 & 26 & 28 & 30\\
   \hline
   Unidades & 20 & 40 & 48 & 57 & 59 & 64 & 66 & 68 & 69 & 70 & 70 & 70 & 70 & 70 & 70 & 69\\
	\hline
   Mes  & 32 & 34 & 36 & 38 & 40 & 42 & 44 & 46 & 48 & 50 & 52 & 54 & 56 & 58 & 60 \\
	\hline
   Unidades  & 69 & 68 & 67 & 65 & 63 & 60 & 57 & 54 & 50 & 47 & 42 & 39 & 33 & 30 & 26 \\
\end{tabular}


\end{document}
